\documentclass{beamer}
\usetheme{Madrid}
\usecolortheme{seahorse}

\title{Prompt Engineering}
\author{Tú Nombre}
\date{\today}

\begin{document}

\begin{frame}
  \titlepage
\end{frame}

\begin{frame}{Índice}
  \tableofcontents
\end{frame}

\section{Introducción}
\begin{frame}{Introducción}
  \begin{itemize}
    \item ¿Qué es el \textit{prompt engineering}?
    \item Importancia en el procesamiento del lenguaje natural.
  \end{itemize}
\end{frame}

\section{Definición}
\begin{frame}{Definición}
  \begin{block}{Prompt Engineering}
    Es el proceso de diseñar y optimizar de manera cuidadosa las instrucciones o consultas (\textit{prompts}) para obtener mejores resultados en modelos de lenguaje y sistemas de inteligencia artificial.
  \end{block}
\end{frame}

\section{Técnicas de Prompt Engineering}
\begin{frame}{Técnicas de Prompt Engineering}
  \begin{enumerate}
    \item Afinación de prompts.
    \item Generación de prompts.
    \item Evaluación de prompts.
  \end{enumerate}
\end{frame}

\section{Ejemplos}
\begin{frame}{Ejemplos}
  \begin{itemize}
    \item Ejemplo 1: Aumento de datos con prompts.
    \item Ejemplo 2: Mejora de la coherencia en respuestas.
  \end{itemize}
\end{frame}

\section{Conclusiones}
\begin{frame}{Conclusiones}
  \begin{itemize}
    \item El prompt engineering es esencial para maximizar el rendimiento de modelos de lenguaje.
    \item Impacto en aplicaciones prácticas.
  \end{itemize}
\end{frame}

\begin{frame}{Preguntas}
  \centering
  \Huge ¿Preguntas?
\end{frame}

\end{document}

